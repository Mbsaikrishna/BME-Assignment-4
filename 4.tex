\documentclass[12pt]{article}

\begin{document}

\section*{What is disruptive innovation}

Disruptive innovation is the process through which a smaller company, usually with fewer resources, rises upwards and competes with larger, more established companies.

\section*{What are the benefits of disruptive innovation?}

Disruptive innovation concepts allow businesses to take a step back and evaluate their current products and services, identifying areas that may be improved, consumer requirements that could benefit from a creative solution, and more.
\section*{Who benefits from disruptive innovation?}
Startup enterprises can use disruptive technology to obtain a major foothold in established industries. Those that are the first to offer the new technology might position themselves as thought leaders in a new market.

\section*{Examples}
Following are the examples of disruptive innovation:-

\section{Consumer electronics, wearables, and mobile applications}

When a patient went to the doctor's office in the past, they could only get biometric data regarding their pulse, heart rate, blood oxygen, and blood pressure. Using data from Fitbits, smartwatches, and mobile phone fitness apps, individuals may now take control of their own health journey. Although the large quantity of personal information acquired by these apps has raised legal and ethical concerns about data privacy, physicians can use the data gathered from these wearables to make treatment recommendations.

\section{Artificial intelligence and machine learning}

AI apps can help with patient intake, scheduling, and invoicing. Patients' questions are answered via chatbots. AI can compile and analyse survey results thanks to its natural language processing skills. AI will most likely become more widely used as a means of lowering healthcare expenses and allowing doctors and staff to focus on patient care. The concerns around database management and patient privacy must be understood by healthcare leaders.


\section{Telemedicine}

COVID-19 has unquestionably accelerated telemedicine delivery, and experts agree that telemedicine is here to stay. It works, doctors will get paid for telemedicine consultations, and many patients prefer it. Telemedicine, on the other hand, is heavily reliant on internet connectivity, and some parts of the United States still lack it.

\section{Blockchain}
Blockchain is a database technology that stores data and links it in a secure and usable way using encryption and other security features. Many elements of healthcare are made easier as a result of this invention, including patient records, supply and distribution, and research. With blockchain applications, IT entrepreneurs have joined the healthcare sector, changing how providers handle medical data.

\section{Electronic health records and big data}
Electronic health records (EHRs) have been a growing part of patient care
since the adoption of the Affordable Care Act. The massive amount of
EHR data goes far beyond patient health records, however, and can be
used to conduct research, improve care, build AI applications, and create
new business opportunities. Therefore, healthcare providers have to be
aware of the issues surrounding EHR security











\end{document}